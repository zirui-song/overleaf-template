%===============================================================================
% LAYOUT
%===============================================================================

% DOCUMENT GEOMETRY:
\usepackage{amssymb,amsmath,amsthm} 
\usepackage{color} 
\usepackage[top=1in, bottom=1in, left=1in, right=1in]{geometry}
\linespread{2} % Increases vertical spacing

%\setlength\parindent{0pt} % No indentation
\setlength{\parskip}{0.35\baselineskip} % Vertical space between paragraphs
%\addtolength{\footnotesep}{2mm} % change to 1mm

\usepackage{graphicx,booktabs}
%\usepackage[round]{natbib}
%\usepackage{hyperref,float}
%\usepackage[longnamesfirst]{natbib}
%\usepackage{natbib,anysize}
\usepackage{lscape}
\usepackage{setspace}
\usepackage{subcaption}
\usepackage{stmaryrd}
\usepackage{comment}
\usepackage{mathrsfs}
\usepackage{setspace}
\usepackage{titlesec}
\usepackage{longtable}
\usepackage{tocloft}
\usepackage[font=sc,labelsep=none]{caption}
\usepackage{multirow}
\usepackage{footmisc}
\usepackage{arydshln}
\usepackage{subfigure}

% FONT:
\usepackage[utf8]{inputenc}
	% Encoding of input text
\usepackage[T1]{fontenc}
	% Encoding of fonts
\usepackage{lmodern}
	% Latin modern font - needed for fontenc
\usepackage{crimson}
\usepackage[kerning]{microtype}
	% Better looking text
\usepackage[english]{babel}
	% Language control, for hyphenation etc

% COLORS:
\usepackage[dvipsnames]{xcolor}
	% expanded colour palette
\usepackage{tcolorbox}
	% Introduces the \newcolorbox{}{} and \begin{tcolorbox} commands.
\newtcolorbox{mybox}{colback=red!5!white,colframe=red!75!black}
	% customise the text boxes

% COMMENTS:
% Make coloured comments in text like \comment{This is a coloured comment.}
\renewcommand{\comment}[1]{{\bfseries \color{Green} #1}}
% \newcommand{\comment}[1]{}  % Uncomment this line to  h i d e  all \comments{}

% LINKS AND NAVIGATION:
\usepackage{xr-hyper}
\usepackage[hidelinks,colorlinks,urlcolor=NavyBlue,linkcolor=NavyBlue,citecolor=NavyBlue]{hyperref}

\makeatletter
\newcommand*{\addFileDependency}[1]{% argument=file name and extension
  \typeout{(#1)}% latexmk will find this if $recorder=0 (however, in that case, it will ignore #1 if it is a .aux or .pdf file etc and it exists! if it doesn't exist, it will appear in the list of dependents regardless)
  \@addtofilelist{#1}% if you want it to appear in \listfiles, not really necessary and latexmk doesn't use this
  \IfFileExists{#1}{}{\typeout{No file #1.}}% latexmk will find this message if #1 doesn't exist (yet)
}
\makeatother

\newcommand*{\myexternaldocument}[1]{%
    \externaldocument{#1}%
    \addFileDependency{#1.tex}%
    \addFileDependency{#1.aux}%
}

\myexternaldocument{InternetAppendix}

\usepackage{bookmark}
	% pdf bookmarks

% HEADER AND FOOTER:
\usepackage{fancyhdr}
\rhead{}
\cfoot{\thepage}
\pagestyle{plain} % "fancy" for line at page head
\fancyhf{}

% FIGURES AND TABLES:
\usepackage[section]{placeins}
	% Keep floats (figures, tables) within the section
\usepackage{booktabs}
	% Better looking tables, also see Fear (2005) for usage and style rules
\usepackage{graphicx}
\usepackage{float}
	% For easier figure placement
%\usepackage{subcaption}
	% Package for several figures in one environment


%===============================================================================
% BIBLIOGRAPHY
%===============================================================================

%\usepackage[utf8]{inputenc}
%\usepackage[english]{babel}
\usepackage[authoryear]{natbib}


% remove the number label before reference


%===============================================================================
% MATH TYPE SETTING OPTIONS
%===============================================================================

%\usepackage{amsmath,amssymb,amsfonts,mathrsfs,accents}
\DeclareMathOperator*{\argmin}{argmin}
\usepackage{bm}
	% Bold and italic mathsymbols AND letters
\usepackage{dsfont}
	% nice indicator function with \mathds{1}


\newtheorem{theorem}{Theorem}
\newtheorem{proposition}{Proposition}
\newtheorem{claim}{Claim}
\newtheorem{axiom}{Axiom}
\newtheorem{condition}{Condition}
\newtheorem{conjecture}{Conjecture}
\newtheorem{corollary}{Corollary}
\newtheorem{definition}{Definition}
\newtheorem{assumption}{Assumption}
\newtheorem{lemma}{Lemma}
\newtheorem{property}{Property}
\newtheorem{fact}{Fact}
\newtheorem{example}{Example}
\newtheorem{observation}{Observation}
%===============================================================================
% MISC PACKAGES
%===============================================================================

\usepackage{enumerate}
	% package for making different lists
\usepackage[babel]{csquotes}
	% Better looking quotes
\usepackage{nicefrac}
\usepackage{capt-of}
	% Allows caption for 'align' environment, i.g.
\usepackage{indentfirst}

\makeatletter
\renewcommand\paragraph{\@startsection{paragraph}{4}{\z@}%
                                      {\parskip}%{3.25ex \@plus1ex \@minus.2ex}%
                                      {-1em}%
                                      {\normalfont\normalsize\bfseries}}
\makeatother
 
%===============================================================================
% MISC OPTIONS
%===============================================================================

\addto\captionsenglish{\renewcommand{\contentsname}{Overview}}
	% Rename title of table of contents
%\numberwithin{equation}{section}
	% Uncomment this if you want to number equations by section
%\renewcommand{\thesubsection}{\thesection.\alph{subsection}}
	% Uncomment this if you want subsection in letters

\newif\ifdraft % to declare sections of the documents to hide conditional on
	% being in draft mode

% ROMAN NUMBERS FOR QJE
\renewcommand{\thetable}{\arabic{table}}
\renewcommand{\thefigure}{\arabic{figure}}

\renewcommand{\thesection}{\arabic{section}} 
\renewcommand{\thesubsection}{\arabic{section}.\arabic{subsection}}

\makeatletter
\newcommand*\ExpandableInput[1]{\@@input#1 }
\makeatother
\newcommand{\sym}[1]{\rlap{#1}}% Thanks to David Carlisle
